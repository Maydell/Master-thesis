\section{Background}
This thesis will cover the application of unsupervised learning to determine roles of players in esports teams. Esport, or electronic sport, is competition in video games. Most of the currently popular games played as esports are played as competitions between either two teams or two players. The computer game used for this thesis is {\it Counter-strike: Global Offensive} (henceforth referred to as CS:GO), which will be described further in the next section. The game is played by two teams competing against each other, with players filling different roles in the gameplay. Sometimes there are roles defined by the game, and sometimes there are roles set by the team as part of the game plan (or tactics). What this thesis will try to achieve is to find these roles in the data from played matches using unsupervised learning.

\subsection{CS:GO}
In CS:GO, two teams of five players compete by using guns and other utilities to complete objectives and/or ``eliminating'' the opposing teams characters in the game. CS:GO is a {\it first person shooter} (FPS), referring to the positioning of the camera in the game being first person and the usage of guns. The game is played in rounds, with each won round giving the winning team 1 point, and the game ending when one team has received 16 points. Draws are possible and will usually be handled by overtime, meaning the winning score will be increased from 16. The teams each play as one of two sides: the {\it Terrorists}, or the {\it Counter-terrorists}. The sides are swapped when 16 rounds have been played. The objectives of each round are dependent on the side. The {\it Terrorist} side has the objective to plant a bomb at one of several designated positions on the playing field, and subsequently to guard the planted bomb during a limited time until it explodes. If the bomb explodes, or the team ``eliminates'' the opposing team's characters, the {\it Terrorist} side wins the round. The {\it Counter-terrorist} side, consequently, has the main objective to prevent the bomb from being planted. If the bomb gets planted, the objective becomes to prevent if from exploding by defusing it in time. If they manage to ``eliminate'' the opposing team's characters before the bomb gets planted {\it or} if they defuse the bomb before it explodes, the {\it Counter-terrorist} side wins the round.

At the beginning of each round each player receives money to be spent on weapons and utility. The amount of money each player receives depends on their performance in the previous round, if the player's team won the previous round or how many rounds in a row the player's team has lost, amongst other factors.

The roles in CS:GO are not decided by the game, but rather by the team and/or circumstance. The role a player fulfills depends on, for instance, what side the team is currently playing on, how much money the player has to spend on weapons and utility, how much money the other players in their team have, and what weapon they specialize in playing with. These are just a few of the factors that a team captain can use to decide how the team should play a round. As some of them can change with each round, players will often change their role from one round to the next. There are, however, roles that are invariant to changes in economy (such as an aggressive playing style), and roles that a player will fulfill any time during the game when the economy allows for it (such as playing with a very expensive and powerful weapon).

\subsection{The data}
CS:GO game servers record the network communication between the server and all the clients in a {\it demo} file. This file can be used to replay the match at a later time through the game client. Abios Gaming downloads all available demo files from professional matches and extracts most of the data contained in them. The data is inserted into a relational database where it can be queried. Some of the data extracted from the demo files is positions at a regular interval, any event relevant to the above mentioned objectives, the economy of each player, and the status of each objective at the end of each round (i.e. who won the round and which objective was completed).

The use of data from professional matches is important for the purpose of this thesis. The roles of players in professional teams will be more polished by training and should therefor be better defined and players will be more successful in playing according to their role.

\subsubsection{Attributes}
Each datapoint will be in the scope of a single round and for a single player. 


Preliminary sources are the listed in the bibliography. \cite{coates2011analysis} \cite{figueiredo2002unsupervised} \cite{hastie2009unsupervised} \cite{10.1007/978-3-319-24589-8_9}